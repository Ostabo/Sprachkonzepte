%%%%%%%%%%%%%%%%%%%%%%%%%%%%%%%%%%%%%%%%%%%%%%%%%%%%%%%%%%%%%
%% Scriptsprachen
%%%%%%%%%%%%%%%%%%%%%%%%%%%%%%%%%%%%%%%%%%%%%%%%%%%%%%%%%%%%%
\chapter{Aufgabe 7}
\label{sec:aufgabe7}

\subsection*{Aufgabe}
Implementieren Sie eine kleine Anwendung mit einer Scriptsprache und analysieren Sie, welche typischen Eigenschaften einer Scriptsprache Sie dabei ausnutzen.

\textit{Vorschläge:}\newline
\textit{JavaScript in einer Webseite}\newline
\textit{Abfrage eines Webservice mit Python, z.B. Feiertage bei feiertage-api.de oder Wechselkurse bei zoll.de -> Service -> Online-Fachanwendungen ...}

\newline
\subsection*{Lösung}
\newline
\begin{code}[language=javascript, caption={JavaScript}, label={lst:Aufgabe7_js}]
const url = 'https://feiertage-api.de/api/';
const params = '?jahr=2022&nur_land=BW';
let request = url + params

function isValidRequest(url) {
    return !!url.match(/^https?:\/\/.+.(de|com|net)\/\w*\/?(\?\w+=.+(&\D+=.+)*)?$/)
}

if (isValidRequest(request)) {
    const response = fetch(request)
    response
        .then(r => r.json())
        .then(data => {
            console.log(data)
            console.log(data['Karfreitag'])
            request = data;
            console.log(request);
        });
}
\end{code}
\newline

Das Skript fragt die Feiertage für das Jahr 2022 in Baden-Württemberg ab.
Dabei sind einige Eigenschaften einer Skriptsprache zu erkennen.
Zum einen werden Konstanten und Variablen, Datentypen automatisch zugewiesen.
So sind die Variablen \textit{url} und \textit{params} vom Typ \textit{String} - \textit{request} zunächst auch.
Mit der Funktion \textit{isValidRequest} wird überprüft, ob die URL gültig ist.
Dabei wird die URL mit einem regulären Ausdruck überprüft.
Die Methode \textit{match} gibt ein Array zurück, wenn der reguläre Ausdruck mit der URL übereinstimmt - ansonsten \textit{null}.
Mit dem Doppel-Not-Operator \textit{(!!)} wird das Ergebnis auf \textit{true} oder \textit{false} umgewandelt.
Hier kommt die Eigenschaft der dynamischen Typisierung zum Tragen.
\newline
Sollte die URL gültig sein, wird eine Anfrage an die URL, mit der Methode \textit{fetch}, gesendet.
Die Antwort wird in der Variable \textit{response} gespeichert. Auch hier wird die dynamische Typisierung genutzt.
Die Antwort ist vom Typ \textit{Promise}.
Mit der Methode \textit{then} wird auf den Promise gewartet und die Antwort in der Lambda-Variablen \textit{r} gespeichert.
Das empfangene JSON-Objekt wird in die Variable \textit{data} übergeben und auf die Konsole ausgegeben.
Um auf die einzelnen Elemente des JSON-Objekts zuzugreifen, wird der Schlüssel verwendet.
In diesem Fall ist der Schlüssel der Name des Feiertags.
Hier wird also der String \textit{Karfreitag} als Array-Index verwendet.
Abschließend wird die Variable \textit{request} mit dem JSON-Objekt überschrieben und auf der Konsole ausgegeben.
Somit hat sich der Datentyp von \textit{String} implizit auf \textit{Object} geändert.

\newline
